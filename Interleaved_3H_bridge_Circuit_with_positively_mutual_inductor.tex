%Interleaved three H-Bridge Topology
% Author MOhammad Wasiq

\documentclass [border=2.5pt]{standalone}
\usepackage{tikz}
\usepackage[americanvoltagesource,cuteinductors,smartlabels]{circuitikz} % A package to draw electrical networks with TikZ
%Design parameters, change it according to your need
\ctikzset{bipoles/thickness=0.5}%change the thickness of the outerlines of the diode
\ctikzset{bipoles/length=0.5cm}%Change the length of the diodes, inductor and the battery
\ctikzset{bipoles/diode/height=0.6}%change the height of the diode
\ctikzset{bipoles/diode/width=0.6}%Varies the width of the diode
\ctikzset{tripoles/thickness=0.5}% varies the thickness of the IGBT, higher values increase boldness of the IGBT
\ctikzset{bipoles/vsourceam/height/.initial=1}
\ctikzset{bipoles/vsourceam/width/.initial=1}
\ctikzset{bipoles/vsourceam/length/.initial=1}
%\ctikzset{V/height=1}%Change the height of battery
% defining distances between elements--- One can change the dimension for the converter by changing these distances:
\newcommand\IHD{0.9} %Horizontal distance between IGBTs in a cell
\newcommand\IVD{0.4} %Vertical distance between IGBTs in a cell
\newcommand\VS{0.45} %distance from the voltage source
\newcommand\VSDA{0.9} %distance from the positive terminal of the voltage source
\newcommand\VSDB{0.2}%distance from the negative terminal of the voltage source
\newcommand\DAR{-2.5}%Distance between the arrays
\newcommand\LDP{0.5}
\newcommand\LDN{0.5}
\newcommand\ARC{0.85}
\newcommand\CLP{0.38}
\newcommand\CLN{0.45}
\newcommand\HLP{0.2}
\newcommand\HLN{0.2}
\newcommand\VLP{0.74}
\newcommand\VLN{-0.64}
\newcommand\FDP{0.375}
\newcommand\FDN{0}
\newcommand\DP{0.7}
\newcommand\DN{0.625}

\newcommand\fillsize{0.5} %fillsize
\tikzstyle{every node}=[font=\tiny]%change the fontsize, change it to \large and see the changes
\tikzstyle{every path}=[line width=0.4 pt, line cap=round, line join=round]%Change the thicknes of the wire
\begin{document}
\begin{circuitikz}[american voltages]
%%%%%%%%%%%%%%%%%%%%%%%%%%%%%%%%%%%%%%%%%%%%%%%%%%%%%%%%%%%%%%%%%%%%%%%%%%%%%%%%%%%%%%%%%%%%%%%%%%%%%%%%%%
%%%%%%%%%%%%%%%%%%%%%%%%%%%%%%%%%%%%%%%%%  Array NO.1  %%%%%%%%%%%%%%%%%%%%%%%%%%%%%%%%%%%%%%%%%%%%%%%%%%%
%%%%%%%%%%%%%%%%%%%%%%%%%%%%%%%%%%%%%%%%%%%%%%%%%%%%%%%%%%%%%%%%%%%%%%%%%%%%%%%%%%%%%%%%%%%%%%%%%%%%%%%%%%

%%%%%%%%%%%%%%%%%%%%%%%%%%%%%%%%%%%%%%%%%%%%%%%%%%%%%%%%%%%%%%%%%%%%%%%%%%%%%%%%%%%%%%%%%%%%%%%%%%%%%%%%%%
%%%%%%%%%%%%%%%%%%%%%%%%%%%%%%%%%%%%%%%%%  Parrallel CELL NO.1  %%%%%%%%%%%%%%%%%%%%%%%%%%%%%%%%%%%%%%%%%%
%%%%%%%%%%%%%%%%%%%%%%%%%%%%%%%%%%%%%%%%%%%%%%%%%%%%%%%%%%%%%%%%%%%%%%%%%%%%%%%%%%%%%%%%%%%%%%%%%%%%%%%%%%
%---- Cell 1
%B2A.C means from node B to node A and connected to the collector terminal of the IGBT
\coordinate  (AB) at (0,0); %%AB stands for phase A, Bottom B
\draw (AB) node[nigbt, bodydiode,anchor=E,red,text=black](B4A){$S_{2b}$} --++(-\IHD,0) node[nigbt, bodydiode, anchor=E,red,text=black](B2A){$S_{1b}$} (B2A.E); % using "node" we defined IGBT with diode 
\draw (B2A.C) --++(0,\IVD) node[nigbt, bodydiode, anchor=E,red,text=black](B1A){$S_{1a}$} ; %% B1A stands for Bottom 1st phase A 
\draw (B4A.C) --++(0,\IVD) node[nigbt, bodydiode,anchor=E,red,text=black](B3A){$S_{2a}$} ;
\draw (B3A.C) -- (B1A.C){};

%Connection of the H-bridge converter to the Battery
\draw (B1A.C) --++(-\VS-0.5,0)--++(-0.5,0) coordinate (ABVP); %% positive pole of the battery
\draw (B2A.E) --++(-\VS-0.5,0)--++(-0.5,0) coordinate (ABVN); %%  negative pole of the battery


%Connection of the output terminal to the filter
\draw  (B2A.C)++ (0,\FDP) coordinate  (Con1AB) ; %% Con1AB stands for output connection from leg A to the inductor
\filldraw  (Con1AB) circle (\fillsize pt);

\draw  (B4A.C)++ (0,\FDN) coordinate  (Con2AB) ; %% Con1AB stands for output connection from leg B to the inductor
\filldraw  (Con2AB) circle (\fillsize pt);
%The arc on the wire
\draw (Con1AB) --++ (\ARC,0)  arc (180:0:0.06) -- ++(\CLP,0) coordinate(ABOP); %ABOP is the positive O/P point
\draw (Con2AB) --++ (\CLN,0)  coordinate (ABON); %ABON is the negative O/P point
\draw (ABOP)--++(\HLP,0)coordinate(ABCP);
\draw (ABCP) to [thick,L,color=blue,l = $L_1$] ++ (\LDP,0) coordinate(ABEP);%End Cordinte of the first cell is ABEP an ABEN
\draw [fill=black] (\DP,\DN)  circle (0.5pt);% Dots for Negatively coupled mutual inductor
\draw [fill=black] (\DP,\DN+0.375)  circle (0.5pt);
\draw (ABON)--++ (\HLP,0) to [thick,L,color=blue,l_ = $L_2$] ++ (\LDN,0) coordinate(ABEN);

\draw (ABEP)--++ (0,\VLP-0.025)arc (270:90:0.06)--++(0,\HLP-0.1)coordinate(MidPA);
\draw (MidPA)--++(\IHD+1.5,0)coordinate(MidPB);
\draw (MidPB)--++(0,-\VLP-0.19)coordinate(CDEP);%Connection to the NExt Cell 2
\draw (ABEN)--++ (0,\VLN-0.05) arc (90:270:0.06)--++(0,-\HLN+0.1)coordinate(MidNA);
\draw (MidNA)--++(\IHD+1.5,0)coordinate(MidNB);
\draw (MidNB)--++(0,-\VLN+0.27)coordinate(CDEN);

%%%%%%%%%%%%%%%%%%%%%%%%%%%%%%%%%%%%%%%%%%%%%%%%%%%%%%%%%%%%%%%%%%%%%%%%%%%%%%%%%%%%%%%%%%%%%%%%%%%%%%%%%%
%%%%%%%%%%%%%%%%%%%%%%%%%%%%%%%%%%%%%%%%%   Parrallel CELL NO.2  %%%%%%%%%%%%%%%%%%%%%%%%%%%%%%%%%%%%%%%%%
%%%%%%%%%%%%%%%%%%%%%%%%%%%%%%%%%%%%%%%%%%%%%%%%%%%%%%%%%%%%%%%%%%%%%%%%%%%%%%%%%%%%%%%%%%%%%%%%%%%%%%%%%%

% Second cell placement 
\coordinate  (CD) at (2.4,0); %%AB stands for phase A, Bottom B
\draw (CD) node[nigbt, bodydiode, anchor=E,red,text=black](D4C){$S_{2b}$} --++(-\IHD,0) node[nigbt, bodydiode, anchor=E,red,text=black](D2C){$S_{1b}$} (D2C.E); % using "node" we defined IGBT with diode 
\draw  (D2C.C) --++(0,\IVD) node[nigbt, bodydiode, anchor=E,red,text=black](D1C){$S_{1a}$} ; %% B1A stands for Bottom 1st phase A 
\draw  (D4C.C) --++(0,\IVD) node[nigbt, bodydiode,anchor=E,red,text=black](D3C){$S_{2a}$} ;
\draw  (D3C.C) -- (D1C.C){};

%BAttery connection
\draw (D1C.C)--++(-\VSDB,0)--++ (0,\HLP)--++ (-\IHD-1.75,0)--++(0,-\HLP) coordinate (ABCVP); %% positive pole of the battery
\filldraw  (ABCVP) circle (\fillsize pt);
\draw (D2C.E)--++(-\VSDB,0)--++ (0,-\HLN)--++(-\IHD-1.75,0)--++(0,\HLN) coordinate (ABDVN); %%  negative pole of the battery
\filldraw  (ABDVN) circle (\fillsize pt);

%Connection of the output terminal to the filter
\draw  (D2C.C)++ (0,\FDP) coordinate  (Con1CD) ; %% Con1AB stands for output connection from leg A to the inductor
\filldraw  (Con1CD) circle (\fillsize pt);

\draw  (D4C.C)++ (0,\FDN) coordinate  (Con2CD) ; %% Con1AB stands for output connection from leg B to the inductor
\filldraw  (Con2CD) circle (\fillsize pt);
%The arc on the wire
\draw (Con1CD) --++ (\ARC,0)  arc (180:0:0.06) -- ++(\CLP,0) coordinate(CDOP); %ABOP is the positive O/P point
\draw (Con2CD) --++ (\CLN,0)  coordinate (CDON); %ABON is the negative O/P point
\draw (CDOP)--++ (\HLP,0) to [thick,L,color=blue,l = $L_1$] ++ (\LDP,0)coordinate(CDEP);%Connection to the Resistor and defining the Coordinate E1
\draw (CDON)--++ (\HLP,0) to [thick,L,color=blue,l_ = $L_2$] ++ (\LDN,0)coordinate(CDEN);
\draw [fill=black] (\DP+2.4,\DN)  circle (0.5pt);% Dots for Negatively coupled mutual inductor
\draw [fill=black] (\DP+2.4,\DN+0.375)  circle (0.5pt);

\draw (MidPB)--++(\IHD+1.5,0)--++(0,-\VLP-0.19)coordinate(EFEP);%Connection to the NExt Cell 3
\filldraw  (MidPB) circle (\fillsize pt);
\draw (MidNB)--++(\IHD+1.5,0)--++(0,-\VLN+0.27)coordinate(EFEP);
\filldraw  (MidNB) circle (\fillsize pt);

%%%%%%%%%%%%%%%%%%%%%%%%%%%%%%%%%%%%%%%%%%%%%%%%%%%%%%%%%%%%%%%%%%%%%%%%%%%%%%%%%%%%%%%%%%%%%%%%%%%%%%%%%
%%%%%%%%%%%%%%%%%%%%%%%%%%%%%%%%%%%%%%%   Parrallel CELL NO.3   %%%%%%%%%%%%%%%%%%%%%%%%%%%%%%%%%%%%%%%%%
%%%%%%%%%%%%%%%%%%%%%%%%%%%%%%%%%%%%%%%%%%%%%%%%%%%%%%%%%%%%%%%%%%%%%%%%%%%%%%%%%%%%%%%%%%%%%%%%%%%%%%%%%

% Second cell placement 
\coordinate  (EF) at (4.8,0); %%AB stands for phase A, Bottom B
\draw (EF) node[nigbt, bodydiode, anchor=E,red,text=black](F4E){$S_{2b}$} --++(-\IHD,0) node[nigbt, bodydiode, anchor=E,red,text=black](F2E){$S_{1b}$} (F2E.E); % using "node" we defined IGBT with diode 
\draw  (F2E.C) --++(0,\IVD) node[nigbt, bodydiode, anchor=E,red,text=black](F1E){$S_{1a}$} ; %% B1A stands for Bottom 1st phase A 
\draw  (F4E.C) --++(0,\IVD) node[nigbt, bodydiode,anchor=E,red,text=black](F3E){$S_{2a}$} ;
\draw  (F3E.C) -- (F1E.C){};

%BAttery connection
\draw (F1E.C)--++(-\VSDB,0)--++ (0,\HLP)--++(-0.09,0) arc (0:180:0.06)--++(-\IHD-1.29,0)  coordinate (ABCDP); %% positive pole of the battery
\filldraw  (ABCDP) circle (\fillsize pt);
\draw (F2E.E)--++(-\VSDB,0)--++ (0,-\HLN)--++(-0.09,0) arc (0:180:0.06)--++(-\IHD-1.29,0) coordinate (ABCDN); %%  negative pole of the battery
\filldraw  (ABCDN) circle (\fillsize pt);

%Connection of the output terminal to the filter
\draw  (F2E.C)++ (0,\FDP) coordinate  (Con1EF) ; %% Con1AB stands for output connection from leg A to the inductor
\filldraw  (Con1EF) circle (\fillsize pt);

\draw  (F4E.C)++ (0,\FDN) coordinate  (Con2EF) ; %% Con1AB stands for output connection from leg B to the inductor
\filldraw  (Con2EF) circle (\fillsize pt);
%The arc on the wire
\draw (Con1EF) --++ (\ARC,0)  arc (180:0:0.06) -- ++(\CLP,0) coordinate(EFOP); %ABOP is the positive O/P point
\draw (Con2EF) --++ (\CLN,0)  coordinate (EFON); %ABON is the negative O/P point
\draw (EFOP)--++ (\HLP,0) to [thick,L,color=blue,l = $L_1$] ++ (\LDP,0)coordinate(EFEP);
\draw (EFON)--++ (\HLP,0) to [thick,L,color=blue,l_ = $L_2$] ++ (\LDN,0)coordinate(EFEN);
\draw [fill=black] (\DP+4.8,\DN)  circle (0.5pt);% Dots for Negatively coupled mutual inductor
\draw [fill=black] (\DP+4.8,\DN+0.375)  circle (0.5pt);

%BAttery
\tiny
\draw (ABVP) to [thick,V,color=violet,l_=${V_{dc}}$] (ABVN); %Battery Connection

%Load
\draw (EFEP) -++(0.3,0) to [thick, R,color = red,l = $R$] ++(\LDP+0.3,0) to [thick,L,color=blue,l = $L_{L}$]++ (\LDP+0.1,0)coordinate (E1);;%Connection to the Resistor and defining the Coordinate E1
\draw (E1)|-(EFEN);%End terminal connection
\filldraw  (EFEP) circle (\fillsize pt);
\filldraw  (EFEN) circle (\fillsize pt);
\end{circuitikz}
\end{document}