%Interleaved three H-Bridge Topology
% Author MOhammad Wasiq

\documentclass [border=2.5pt]{standalone}
\usepackage{rotating}
% \usepackage{xcolor}
\usepackage{transparent}

\usepackage{pgf,tikz}
\usetikzlibrary{positioning,fit,calc}
\usepackage[american,cuteinductors,smartlabels]{circuitikz} % A package to draw electrical networks with TikZ
%Design parameters, change it according to your need
\ctikzset{bipoles/thickness=0.5}%change the thickness of the outerlines of the diode
\ctikzset{bipoles/length=0.5cm}%Change the length of the diodes, inductor and the battery
\ctikzset{bipoles/diode/height=0.6}%change the height of the diode
\ctikzset{bipoles/diode/width=0.6}%Varies the width of the diode
\ctikzset{tripoles/thickness=0.5}% varies the thickness of the IGBT, higher values increase boldness of the IGBT
\ctikzset{bipoles/vsourceam/height/.initial=1}
\ctikzset{bipoles/vsourceam/width/.initial=1}
\ctikzset{bipoles/vsourceam/length/.initial=1}
%\ctikzset{V/height=1}%Change the height of battery
% defining distances between elements--- One can change the dimension for the converter by changing these distances:
\newcommand\dimI{7} 
\newcommand\dimII{2} 
\newcommand\dimIII{1.5}
\newcommand\dimIV{3}

\tikzstyle{every node}=[font=\small]%change the fontsize, change it to \large and see the changes
\tikzstyle{every path}=[line width=0.4 pt, line cap=round, line join=round]%Change the thicknes of the wire
\begin{document}
\begin{circuitikz}[american voltages]
% Generator
\tiny
\coordinate (A) at (0,0);
\draw[thick] (A) to [thick,V,color=red,l=${V_{s}}$]++(-0.5,0)coordinate(p1);
\draw[thick] (A)--++(\dimI,0)coordinate(p2);
\draw[thick] (p2)--++(\dimII-0.35,0)coordinate(p);
\draw[thick] (p2)--++(0,-\dimIII-0.25) coordinate(p3);
\draw[thick] (p3) --++(0,-\dimIII-0.25) coordinate(p4);
\draw[thick] (p4) --++(-\dimII+0.35,0) coordinate(p5);
\draw[thick] (p5) --++(-\dimIII-0.75,0) coordinate(p6);
\draw[thick] (p6) --++(-\dimII+0.35,0) coordinate(p7);
\draw[thick] (p7) --++(0,\dimIII+0.25) coordinate(p8);
\draw[thick] (p8) --++(0,\dimIII+0.25) coordinate(p9);
\fill[black] (p9) circle (0.05 cm);
\fill[black] (p2) circle (0.05 cm);
% Currents arrow

% \draw (I12) -- (I13);
% %Load
\node[rectangle,
    draw=black,thick,text=blue,
    fill = white,opacity=1,
    minimum width = 1cm, 
    minimum height = 1.5cm] (r.center) at (p) {\rotatebox[origin=c]{0}{\parbox{1.5cm}{Non-linear \centering{Load}}}};
\node[rectangle,
    draw=black,thick,text=blue,
    fill = white,opacity=1,
    minimum width = 1cm, 
    minimum height = 1.5cm] (r.center) at (p3) {\rotatebox[origin=c]{0}{\parbox{1.5cm}{Harmonic Current  \centering{Detection}}}};
\node[rectangle,
    draw=black,thick,text=blue,
    fill = white,opacity=1,
    minimum width = 1cm, 
    minimum height = 1.5cm] (r.center) at (p5) {\rotatebox[origin=c]{0}{\parbox{1.5cm}{Current Tracking  \centering{Control}}}};
\node[rectangle,
    draw=black,thick,text=blue,
    fill = white,opacity=1,
    minimum width = 1cm, 
    minimum height = 1.5cm] (r.center) at (p6) {\rotatebox[origin=c]{0}{\parbox{1.5cm}{PWM \centering{Generation}}}};
\node[rectangle,
    draw=black,thick,text=blue,
    fill = white,opacity=1,
    minimum width = 1cm, 
    minimum height = 1.5cm] (r.center) at (p8) {\rotatebox[origin=c]{0}{\parbox{1.5cm}{Power \centering{Circuit}}}}; 
\node(r1)[rectangle,
    draw=black,dashed,thick,
    fill = none,opacity=1,
    minimum width = 7.4cm, 
    minimum height = 3.5cm] (r.center) at (4.225,-2.65) {{\parbox{1.5cm}{}}};
\node (r1Label)[text=blue,above=-2.75 of r1] {\textbf{SAPF}};
\draw[-{Latex[length=1.75mm]},red] (0.8,0)--node[xshift=0cm,yshift=-0.25cm,color=black]{\textbf{$I_{s}$}}(0.9,0);
\draw[-{Latex[length=1.75mm]},red] (4.3,0)--node[xshift=0cm,yshift=-0.25cm,color=black]{\textbf{$I_{l}$}}(4.4,0);
\draw[-{Latex[length=1.75mm]},red] (1.45,-0.5)--node[xshift=0.4cm,yshift=-0.1cm,color=black]{\textbf{$I_{inj}$}}(1.45,-0.4);
\draw[-{Latex[length=1.75mm]},red] (1.45,-3.1)--node[xshift=0.3cm,yshift=-0.1cm,color=black]{\textbf{$I_{g}$}}(1.45,-3);
\draw[-{Latex[length=1.75mm]},red] (7,-3)--node[xshift=0.3cm,yshift=-0.1cm,color=black]{\textbf{$I_{h}$}}(7,-3);
\end{circuitikz}
\end{document}
